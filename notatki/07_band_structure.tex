\section{Metody obliczania struktury pasmowej / Rozwiązywania równań Kohna-Shama}


\subsection{PW}
~\\
Rozwiązywanie równań w bazie fal płaskich:
\begin{equation}
\bra{\boldsymbol{r}}\ket{\boldsymbol{k}+\boldsymbol{G}} =
\frac{1}{\sqrt{V}} e^{i\left(\boldsymbol{k}+\boldsymbol{G}\right) \cdot \boldsymbol{r}}
\end{equation}
gdzie $\boldsymbol{G}$ wektor sieci odwrotnej spefłniający warunek $\frac{\hbar^2}{2m_e}\vert\boldsymbol{k}+\boldsymbol{G}\vert^2 \leq E_{\text{cut}}$
\subsection{OPW}
~\\


\subsection{APW}
~\\


\subsection{LCAO}
~\\
Szukamy rozwiązania w kostaci funkcji Blocha.
$$
\Psi_{n\boldsymbol{k}}\left(\boldsymbol{r}\right) = 
\frac{1}{\sqrt{N}} \sum_{\mu} c_{n\mu}\left(\boldsymbol{k}\right) \Phi_{\boldsymbol{k}\mu}\left(\boldsymbol{r}\right)
$$
Funkcja LCAO:
$$
\Phi_{\boldsymbol{k}\mu}\left(\boldsymbol{r}\right) = \frac{1}{\sqrt{N}} \sum_j e^{i\boldsymbol{k}\cdot\boldsymbol{R}_j} R_{\mu l}(\vert\boldsymbol{r}-\boldsymbol{R}_j\vert) Y_{lm}\left(\theta,\phi\right)
$$
Razem funkcję falową rozwijamy w bazie:
$$
\Psi_{n\boldsymbol{k}}\left(\boldsymbol{r}\right) = \frac{1}{N} \sum_{\mu} \sum_{j}
c_{n\mu}\left(\boldsymbol{k}\right) e^{i\boldsymbol{k}\cdot\boldsymbol{r}}
\underset{\text{orbitale atomowe}}{\underbrace{
		R_{\mu l}(\vert\boldsymbol{r}-\boldsymbol{R}_j\vert) Y_{lm}\left(\theta,\phi\right)
}}
$$
Funkcje radialne spełniają równanie:
$$
\left[ -\frac{1}{2} \frac{1}{r} \frac{d^2}{d^2r} r + \frac{1}{2} \frac{\ell(\ell-1)}{r^2} + V_{\mu\ell}(r)\right] R_{\mu \ell}(r) = \left(\varepsilon_{\ell} + \delta \varepsilon_{\ell}\right) R_{\mu \ell}(r)
$$
Dla każdego orbitalu musimy określić zasięg bazy, co można zrobić wprowadzając przesunięcie energii $\delta \varepsilon_{\ell}$.

Otrzymana baza nie jest ortonormalna, nie mniej jednak możemy rozwiązać metodą Ritza:
\begin{equation}
\sum_{\nu} H_{\mu\nu} c_{n\mu}\left(\boldsymbol{k}\right) = E_{n}\left(\boldsymbol{k}\right) \sum_{\mu} S_{\mu\nu} c_{n\mu}\left(\boldsymbol{k}\right)
\end{equation}
$\nu$ indeksuje składowe wektorów własnych.

Konstrukcja bazy LCAO:
\begin{itemize}
	\item Minimalny rozmiar bazy – \textbf{„single-$\zeta$” (SZ)} – jedna funkcja radialna na jedną zapełnioną powłokę ze względu na moment pędu
	\item Możliwości zwiększenia rozmiaru bazy – zwiększenie ilości funkcji radialnych \textbf{„multiple-$\zeta$” (DZ,DZP)} lub dodanie orbitali o różnym $\ell$ („polarization”)
\end{itemize}


\newpage
\subsection{Porównanie metod}
~\\
\begin{multicols}{2}
PW:

APW:
\begin{itemize}
	\item połączenie metod PW i LCAO
	\item 
\end{itemize}

OPW:

LCAO:
\begin{itemize}
	\item Używają pseudopotencjałów
	\item Bazą funkcji falowej jest zbiór orbitali atomowych
	\item Opis „zlokalizowany” – w duchu chemii kwantowej
	\item Wysoka wydajność
	\item Problemy ze zbieżnością
	\item Główne cechy bazy: wielkość, zakres orbitali, kształt
	\item Numeryczne orbitale atomowe (NAO) to rozwiązania numeryczne zagadnienia K-S dla izolowanego (pseudo-) atomu
\end{itemize}

\end{multicols}

