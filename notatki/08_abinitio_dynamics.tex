
\section{Dynamika molekularna w ujęciu Ab-Initio}


\subsection{Twierdzenie Hellmana-Feynmanna}

\begin{thm} \label{theorem:HellmanFeynmann}
	Twierdzenie Feynmana-Hellmanna: pochodna energii
	całkowitej po pewnym parametrze jest równa wartości
	średniej pochodnej hamiltonianu po tym samym
	parametrze:
	$$ \frac{\partial E}{\partial \lambda} = \bra{\Psi} \frac{\partial}{\partial \lambda} \hat{H} \ket{\Psi} $$
\end{thm}
\begin{proof}
	???
\end{proof}

\subsection{Obliczanie sił działających na molekuły i relaksacja kryształu}
~\\
Na mocy twd. \ref{theorem:HellmanFeynmann} możemy napisać, że i-ta składowa siły jest równa:
\begin{subequations}
\begin{equation}
	F_i = -\frac{\partial E}{\partial X_i} = -\bra{\Psi} \frac{\partial}{\partial X_i} \hat{H} \ket{\Psi}
\end{equation}
\begin{equation}
	\boldsymbol{F} = -\nabla E = \int_{\mathbb{R}^3}\Psi^*\left(\boldsymbol{r}\right) \nabla H\left(\boldsymbol{r}\right) \Psi\left(\boldsymbol{r}\right) d^3r
\end{equation}
\end{subequations}

Relaksacja kryształu polega na znalezieniu minimum sił. Jest potrzebna do wyznaczenia struktury pasmowej.

\subsection{Dynamika ab-initio vs klasyczna}
~\\
\begin{multicols}{2}
\textbf{Klasyczna}
\begin{itemize}
\item Stały potencjał
\item Brak elektronowych stopni swobody
\item Brak opisu reakcji chemicznych
\item Zakres do ok. 100 $\mathring{A}$
\item Skala czasowa – do ok. 10 ns
\end{itemize}
\textbf{Ab-initio}
\begin{itemize}
\item Potencjał samozgodny
\item Elektronowe stopnie swobody
\item Opis powstawania i zrywania wiązań chemicznych (reakcji chemicznych)
\item Zakres do ok. 20 $\mathring{A}$
\item Skala czasowa – do ok. 100 ps
\end{itemize}
\end{multicols}

\subsection{Dynamika ab-initio Car-Parinello}
~\\

\subsection{Dynamika ab-initio BO}
~\\

\subsection{Obliczanie relacji dyspersyjnych dla fononów}
~\\
Teoria odpowiedzi liniowej – fonony
