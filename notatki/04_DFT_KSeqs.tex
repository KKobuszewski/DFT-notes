
\section{Twierdzenia Hohenberga-Kohna}
\begin{thm}
\textbf{Hohenberga-Kohna}

Istnieje jednoznaczne odwzorowanie pomiędzy potencjał zewnętrznym
$\hat{V}_{ext}=\sum_{i=1}^{N}v_{ext}(\boldsymbol{r})$ a gęstością
cząstek $n_{0}\left(\boldsymbol{r}\right)$ w niezdegenerowanym stanie
podstawowym i tym samym całkowita energia układu jest wtedy jednoznacznym
funkcjonałem gęstości $n_{0}\left(\boldsymbol{r}\right)$ . \end{thm}
\begin{proof}
(\textbf{1. twierdzenia H-K})

Załóżmy, że istnieją dwa potencjały $\hat{V}_{ext}^{(1)}$ i $\hat{V}_{ext}^{(2)}$,
którym odpowiada pojedyncza gęstość cząstek w stanie podstawowym $n_{0}\left(\boldsymbol{r}\right)$
.

Stan $\ket{\Psi^{(1)}}$ odpowiadający $n_{0}\left(\boldsymbol{r}\right)$
jest stanem podstawowym z $\hat{V}_{ext}^{(1)}$, toteż z zasady wariacyjnej
wynika:

\[
\bra{\Psi^{(1)}}\hat{H}^{(1)}\ket{\Psi^{(1)}}<\bra{\Psi^{(1)}}\hat{H}^{(2)}\ket{\Psi^{(1)}}=\bra{\Psi^{(1)}}\hat{H}^{(1)}\ket{\Psi^{(1)}}+\bra{\Psi^{(1)}}\hat{H}^{(2)}-\hat{H}^{(1)}\ket{\Psi^{(1)}}
\]


I analogicznie stan $\ket{\Psi^{(2)}}$ odpowiadający $n_{0}\left(\boldsymbol{r}\right)$
jest stanem podstawowym z $\hat{V}_{ext}^{(2)}$, toteż z zasady wariacyjnej
wynika:

\[
\bra{\Psi^{(2)}}\hat{H}^{(2)}\ket{\Psi^{(2)}}<\bra{\Psi^{(2)}}\hat{H}^{(1)}\ket{\Psi^{(2)}}=\bra{\Psi^{(2)}}\hat{H}^{(2)}\ket{\Psi^{(2)}}+\bra{\Psi^{(2)}}\hat{H}^{(2)}-\hat{H}^{(1)}\ket{\Psi^{(2)}}
\]


Hamiltoniany $\hat{H}^{(1)}$ i $\hat{H}^{(2)}$ różnią się tylko
zewnętrznym potencjałem:

\[
\bra{\Psi^{(1)}}\hat{H}^{(1)}\ket{\Psi^{(1)}}<\bra{\Psi^{(1)}}\hat{H}^{(1)}\ket{\Psi^{(1)}}+\int_{\mathbb{R}^{3}}\left(V_{ext}^{(2)}\left(\boldsymbol{r}\right)-V_{ext}^{(1)}\left(\boldsymbol{r}\right)\right)n_{0}\left(\boldsymbol{r}\right)d^{3}r
\]
\[
\bra{\Psi^{(2)}}\hat{H}^{(2)}\ket{\Psi^{(2)}}<\bra{\Psi^{(2)}}\hat{H}^{(2)}\ket{\Psi^{(2)}}+\int_{\mathbb{R}^{3}}\left(V_{ext}^{(1)}\left(\boldsymbol{r}\right)-V_{ext}^{(2)}\left(\boldsymbol{r}\right)\right)n_{0}\left(\boldsymbol{r}\right)d^{3}r
\]


Dodając stronami:

\[
\bra{\Psi^{(1)}}\hat{H}^{(1)}\ket{\Psi^{(1)}}+\bra{\Psi^{(2)}}\hat{H}^{(2)}\ket{\Psi^{(2)}}<\bra{\Psi^{(1)}}\hat{H}^{(1)}\ket{\Psi^{(1)}}+\bra{\Psi^{(2)}}\hat{H}^{(2)}\ket{\Psi^{(2)}}
\]


lub 
\[
0<\int_{\mathbb{R}^{3}}\left(V_{ext}^{(2)}\left(\boldsymbol{r}\right)-V_{ext}^{(1)}\left(\boldsymbol{r}\right)+V_{ext}^{(1)}\left(\boldsymbol{r}\right)-V_{ext}^{(2)}\left(\boldsymbol{r}\right)\right)n_{0}\left(\boldsymbol{r}\right)d^{3}r=0
\]


Co jest wewnętrznie sprzeczne!

Stąd wniosek, że początkowe założenie jest nieprawdziwe, więc musi
istnieć tylko jeden $\hat{V}_{ext}$ odpowiadający dokładnie jednej
$n_{0}\left(\boldsymbol{r}\right)$.\end{proof}
\begin{thm}
\textbf{Hohenberga-Kohna}

The functional that delivers the ground state energy of the system,
gives the lowest energy if and only if the input density is the true
ground state density.

\textbf{Uwaga}

Gęstość \textgreek{r} 0 minimalizująca całkowitą energię jest dokładną
gęstością stanu podstawowego. A więc, dla próbnej gęstości (nieujemnej
i całkującej się do N ) zachodzi $E\left[ \tilde{\rho} \right] \leq ­ E\left[ \rho_0 \right] = E_0$ .\end{thm}
\begin{proof}
2. twd. Hohenberga-Kohna

W niezdegenerowanym stanie podstawowym wartość oczekiwana dowolnego
operatora jest funkcjonałem gęstości cząstek minimalizującej energię
stanu podstawowego. $n({\bf r})$ determines $v_{\hbox{ext}}({\bf r})$,
$N$ and $v_{\hbox{ext}}({\bf r})$ determine $\hat{H}$ and therefore
$\Psi$. This ultimately means $\Psi$ is a functional of $n({\bf r})$,
and so the expectation value of $\hat{F}$ is also a functional of
$n({\bf r})$, i.e. ${\displaystyle F[n({\bf r})]=\langle\psi\vert\hat{F}\vert\psi\rangle\,.}$

A density that is the ground-state of some external potential is known
as $v$-representable. Following from this, a $v$-representable energy
functional $E_{v}[n({\bf r})]$ can be defined in which the external
potential $v({\bf r})$ is unrelated to another density $n'({\bf r})$,
\[
E_{v}[n({\bf r})]=\int n'({\bf r})\,v_{\hbox{ext}}({\bf r})\,d{\bf r}+F[n'({\bf r})]\,
\]


and the variational principle asserts, 
\[
{\displaystyle \langle\psi'\vert\hat{F}\vert\psi'\rangle+\langle\psi'\vert\psi\rangle+\langle\psi\vert\hat{V}_{\hbox{ext}}\vert\psi\rangle\thinspace,}
\]
 

where $\psi$ is the wavefunction associated with the correct groundstate
$n({\bf r})$. This leads to, 
\[
{\displaystyle \int n'({\bf r})\,v_{\hbox{ext}}({\bf r})\,d{\bf r}+F[n'(......;\int n({\bf r})\,v_{\hbox{ext}}({\bf r})\,d{\bf r}+F[n({\bf r})]\,,}
\]


and so the variational principle of the second Hohenberg-Kohn theorem
is obtained, 
\[
\ensuremath{{\displaystyle E_{v}[n'({\bf r})]\,>\,E_{v}[n({\bf r})]\,.}}
\]


Although the Hohenberg-Kohn theorems are extremely powerful, they
do not offer a way of computing the ground-state density of a system
in practice. About one year after the seminal DFT paper by Hohenberg
and Kohn, Kohn and Sham {[}9{]} devised a simple method for carrying-out
DFT calculations, that retains the exact nature of DFT. This method
is described next. 
\end{proof}

\subsection{Konsekwencje twierdzeń H-K}
~\\
\begin{enumerate}
	\item Zmiana w zewnętrznym potencjale pociąga za sobą zmianę w gęstości rozkładu cząstek
	$$ \int_{\mathbb{R}^3} \delta v_{ext}\left(\boldsymbol{r}\right) \delta n\left(\boldsymbol{r}\right) d^3r > 0 $$
	\item Dla systemu Coulombowskiego wszystkie parametry układu mogą zostać wyznaczone przy znajomości gęstości elektronowej, tzw. twierdzenie Kato:
	$$ Z_k = - \frac{a_B}{2 n\left(\boldsymbol{r}\right)} \frac{dn\boldsymbol{r}}{d\boldsymbol{r}} \vert_{\boldsymbol{r} \to \boldsymbol{R}_k}$$
\end{enumerate}

\newpage
\section{Metoda Kohna-Shama}


\subsection{Założenia}

Założenia w metodzie Kohna-Shama:
\begin{enumerate}
\item Zastępujemy układ oddziałujacych cząstek w zewnętrznym potencjale
$v_{ext}\left({\bf r}\right)$ układem pomocniczym składającym się
z quasicząstek nieoddziałujących w pewnym efektywnym potencjale $v_{eff}\left({\bf r}\right)$.
\item Zakładamy, że istnieje takie $v_{eff}\left({\bf r}\right)$, które
dokładnie odwzorowuje energię stanu podstawowego układu oddziałującego.
\item Układ nieoddziałujących cząstek jest opisany za pomocą orbitali Kohna-Shama
(pseudofunkcji falowych) $\phi_{i}\left({\bf r}\right)$ (są to rozw.
równania Kohna-Shama).
\end{enumerate}
Przy powyższych założeniach gęstość stanu podstawowego ukł. oddziałującego
jest zadana przez
\[
n\left({\bf r}\right)=\sum_{i=1}^{N}\phi_{i}\left({\bf r}\right)
\]



\subsection{Funkcjonał gęstości energii Kohna-Shama}

Dla hamiltonianu układu wielu ciał:

Na mocy twierdzeń Hohenberga-Kohna można utworzyć funkcjonał gęstości:
\[
E_{HK}\left[n\right]=T\left[n\right]+\int_{\mathbb{R}^{3}}v_{ext}\left({\bf r}\right)d^{3}r+\int_{\mathbb{R}^{3}}v_{int}\left({\bf r}\right)n\left({\bf r}\right)d^{3}r
\]

Wyrażenie na energię całkowitą układu otrzymaną metodą Hartree-Focka można porównać z enegią otrzymaną z funkcjonału Hohenberga-Kohna. Przy założeniach metody Kohna-Shama można dzięki temu przedstawić funkcjonał Hohenberga-Kohna w postaci:

\[
E_{KS}\left[n\right]=T_{S}\left[n\right]+\int_{\mathbb{R}^{3}}v_{ext}\left({\bf r}\right)n\left({\bf r}\right)d^{3}r+E_{H}\left[n\right]+E_{Ion-Ion}\left[n\right]+E_{XC}\left[n\right]
\]

Gęstość elektronową rozbijamy na quasispinorbilate: $n\left(\boldsymbol{r}\right) = \sum_i \vert\phi_i\vert^2\left(\boldsymbol{r}\right)$

$T_{S}\left[n\right]=\int_{\mathbb{R}^{3}}\sum_{i=1}^{N}\frac{\hbar^{2}}{2m}\vert\nabla\phi_{i}\vert d^{3}r$ - funkcjonał energii kinetycznej nieoddziałujących cząstek

$E_{H}\left[n\right]=\frac{e^2}{2}\int_{\mathbb{R}^{3}}\int_{\mathbb{R}^{3}}\frac{n\left(\boldsymbol{r}\right)n\left(\boldsymbol{r}'\right)}{\vert\boldsymbol{r} - \boldsymbol{r}'\vert}\cdot d^{3}r'd^{3}r$ - funkcjonał energii Hartree {[}UZUPELNIĆ{]}

$E_{II}\left[n\right]=
\int_{\mathbb{R}^3} V_{I-I}\left(\boldsymbol{r}\right) n\left(\boldsymbol{r}\right)d^{3}r = \text{const}$ - funkcjonał energii oddziaływania jon-jon

$E_{XC}\left[n\right]=T\left[n\right]+\langle\hat{V}_{int}\rangle-T_{S}\left[n\right]-E_{H}\left[n\right]$
- funkcjonał energii korelacyjno-wymiennej, w ogolności analityczna
postać nie jest znana (gdyby była, to metoda KS byłaby metodą dokładnie odwzorowującą stan podstawowy układu).


\subsection{Równanie Kohna-Shama}
~\\
Nakładając warunek minimalizacji energii przy ustalonej liczbie cząstek możena wyprowadzić równanie Kohna-Shama na quasi-orbital jednocząstkowy:
\begin{subequations}
\begin{equation}
\frac{\delta E_{KS}[n]}{\delta n}\left(\boldsymbol{r}\right) - \mu \int_{\mathbb{R}^3} n\left(\boldsymbol{r}\right) d^3r = \frac{\delta E_{KS}[n]}{\delta n}\left(\boldsymbol{r}\right) - \mu \int_{\mathbb{R}^3} \sum_{j} \vert\phi_j\vert^2 \left(\boldsymbol{r}\right) d^3r = 0
\end{equation}
Stąd po rozseparowaniu na pojedyncze spinorbitale:
\begin{equation}
\left[\right]\phi_i\left(\boldsymbol{r}\right) = \varepsilon_i \phi_i \left(\boldsymbol{r}\right)
\end{equation}
\end{subequations}


\subsection{Hartree-Fock vs Kohn-Sham}


