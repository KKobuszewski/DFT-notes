
\section{Przybliżenia}


\subsection{Przybliżenie adiabatyczne}

Time-dependent adiabatic couplings

Dlaczego oraz kiedy można odseparować skałdniki wolno i szykbo zmienne?


\subsection{Przybliżenie Borna-Oppenheimera}

Polega na zaniedbaniu Time-dependent adiabatic couplings, czyli w
uproszczeniu na zaniedbaniu wolnozmiennych składników przy rozwiązywaniu
r. Schroedingera.

Jeżeli rozseparujemy funkcję falową układu jąder i elektronów w postaci
iloczynu: 
\[
\Psi\left(\boldsymbol{r}_{i},\boldsymbol{R}_{j}\right)=\Phi\left(\boldsymbol{r}_{i};\boldsymbol{R}_{j}\right)\chi\left(\boldsymbol{R}_{j}\right)
\]


To położenia jąder/jonów w przybliżeniu BO mogą być potraktowane jako
parametry układu.

Możemy wtedy rozwiązać równanie na funkcje elektronowe $\Phi$ traktując
położenia ciężkich jąder/rzdzeni atomowych jako parametry. Jest to
tzw. elektronowe równanie Schroedingera:
\[
\hat{H}_{e}^{BO}\Phi\left(\boldsymbol{r}_{i};\boldsymbol{R}_{j}\right)=E_{e}\Phi\left(\boldsymbol{r}_{i};\boldsymbol{R}_{j}\right)
\]
\[
\hat{H}_{e}^{BO}=
\]


Wyprowadzenie: \href{https://en.wikipedia.org/wiki/Born\%E2\%80\%93Oppenheimer_approximation}{https://en.wikipedia.org/wiki/Born-Oppenheimer\_{}approximation}.


