% %%%%%%%%%%%%%%%%%%%%%%!TeX program = lualatex
\documentclass[oneside,polish]{amsart}
\usepackage[T1]{fontenc}
\usepackage[utf8]{inputenc}
\usepackage{lmodern}
\setlength{\parindent}{0bp}
\usepackage[polish]{babel}
\usepackage{float}
\usepackage{amsmath}
\usepackage{amsfonts}
\usepackage{amsbsy}
\usepackage{amstext}
\usepackage{amsthm}
\usepackage{xparse}
\usepackage{physics}
%\usepackage{amssymb}
\usepackage{accents}
\usepackage{esint}
\usepackage[unicode=true]{hyperref}
\usepackage{breakurl}

\makeatletter

%%%%%%%%%%%%%%%%%%%%%%%%%%%%%% LyX specific LaTeX commands.
\DeclareRobustCommand{\greektext}{%
  \fontencoding{LGR}\selectfont\def\encodingdefault{LGR}}
\DeclareRobustCommand{\textgreek}[1]{\leavevmode{\greektext #1}}
\DeclareFontEncoding{LGR}{}{}
\DeclareTextSymbol{\~}{LGR}{126}

%%%%%%%%%%%%%%%%%%%%%%%%%%%%%% Textclass specific LaTeX commands.
\numberwithin{equation}{section}
\numberwithin{figure}{section}
\theoremstyle{plain}
\newtheorem{thm}{\protect\theoremname}

\makeatother

\providecommand{\theoremname}{Twierdzenie}

\begin{document}

\title{Wykład metody ab-intio (DFT)}

\maketitle

\tableofcontents
\newpage

\section{Wielociałowa funkcja falowa}

Wielociałowa funkcja falowa

Katastrofa van Vlecka

Gęstość elektronowa

Hamiltonian układu w problemach ciała stałego:

\[
\hat{H}=\hat{T}_{e}\left(\nabla_{\boldsymbol{r}_{i}}\right)+\hat{T}_{jon}\left(\nabla_{\boldsymbol{R}_{i}}\right)+\hat{V}_{ext}\left(\boldsymbol{r}_{i},\boldsymbol{R}_{j}\right)+\hat{V}_{jon-jon}\left(\boldsymbol{R}_{i}-\boldsymbol{R}_{j}\right)+\hat{V}_{e-jon}\left(\boldsymbol{r}_{i}-\boldsymbol{R}_{j}\right)+\hat{V}_{e-jon}\left(\boldsymbol{r}_{i}-\boldsymbol{r}_{j}\right)
\]

\begin{equation}
\bra{\boldsymbol{r}}\hat{H}\ket{\boldsymbol{r}}=-\frac{\hbar^2}{2 m_e} \sum_{i=1}^{N_e} \nabla_{\boldsymbol{r}_{i}}
-\sum_{I=1}^{N_{jon}} \frac{\hbar^2}{2 M_I} \nabla_{\boldsymbol{R}_{I}}+
\frac{1}{2}\sum_{I=1}^{N_{jon}}\sum_{J=1}^{N_{jon}}\left(\boldsymbol{R}_{i}-\boldsymbol{R}_{j}\right)+
\hat{V}_{e-jon}\left(\boldsymbol{r}_{i}-\boldsymbol{R}_{j}\right)+
\hat{V}_{e-jon}\left(\boldsymbol{r}_{i}-\boldsymbol{r}_{j}\right)
\end{equation}


Rozkłada się na energię kinetyczną elektronów, jonów (najczęściej
jąder/rdzeni atomowych), jednocząstkową energię potencjalną działającą
na cząstki (np. grawitacja) oraz operatory dwucząstkowe przedstawiające
wzajenmne oddziaływania cząstek na siebie.


\newpage


\section{Przybliżenia}


\subsection{Przybliżenie adiabatyczne}

Time-dependent adiabatic couplings

Dlaczego oraz kiedy można odseparować skałdniki wolno i szykbo zmienne?


\subsection{Przybliżenie Borna-Oppenheimera}

Polega na zaniedbaniu Time-dependent adiabatic couplings, czyli w
uproszczeniu na zaniedbaniu wolnozmiennych składników przy rozwiązywaniu
r. Schroedingera.

Jeżeli rozseparujemy funkcję falową układu jąder i elektronów w postaci
iloczynu: 
\[
\Psi\left(\boldsymbol{r}_{i},\boldsymbol{R}_{j}\right)=\Phi\left(\boldsymbol{r}_{i};\boldsymbol{R}_{j}\right)\chi\left(\boldsymbol{R}_{j}\right)
\]


To położenia jąder/jonów w przybliżeniu BO mogą być potraktowane jako
parametry układu.

Możemy wtedy rozwiązać równanie na funkcje elektronowe $\Phi$ traktując
położenia ciężkich jąder/rzdzeni atomowych jako parametry. Jest to
tzw. elektronowe równanie Schroedingera:
\[
\hat{H}_{e}^{BO}\Phi\left(\boldsymbol{r}_{i};\boldsymbol{R}_{j}\right)=E_{e}\Phi\left(\boldsymbol{r}_{i};\boldsymbol{R}_{j}\right)
\]
\[
\hat{H}_{el}^{BO}= \hat{T}_e + \hat{V}_{e-jon} + \hat{V}_{e-e}
\]

Czyli przybliżamy, że $ \langle T_{\text{jon}} \rangle, \langle V_{\text{jon-jon}} \rangle \approx \text{const}$, więc można w ogóle przenieść poziom energii stanu podstawowego tak by była to najniższa wartość własna hamiltonianu elektronowego.

Wyprowadzenie: \href{https://en.wikipedia.org/wiki/Born\%E2\%80\%93Oppenheimer_approximation}{https://en.wikipedia.org/wiki/Born-Oppenheimer\_{}approximation}.



\newpage

\section{Przybliżenie pola średniego i metoda Hartree-Focka}


\subsection{Przybliżenie pola średniego}

~\\
Energię potencjalną oddziaływania elektronów między sobą zastępujemy samouzgodnionym polem średnim
$$ V_{e-e}\left(\boldsymbol{r}_i,\boldsymbol{r}_i\right) = \frac{1}{2}\sum_{i \neq j}^{N_{e}} \frac{ e^2}{\left\vert\boldsymbol{r}_{i}-\boldsymbol{r}_{j}\right\vert} \mapsto U_{scf}\left(\boldsymbol{r}_i\right) = \sum_{j}^{N_{e}} e^2 \int_{\mathbb{R}^3} \frac{ \vert\psi_j\vert^2\left(\boldsymbol{r}'\right) }{\left\vert\boldsymbol{r}_{i}-\boldsymbol{r}'\right\vert} d^3r' $$
W efekcie otrzymujemy tylko operatory jednocząstkowe i równania stają się rozwiązywalne numerycznie.

\subsection{Równania Hartree-Focka}
~\\
Przybliżenie średniego pola dla funkcji falowej w postaci wyznacznika Slatera oraz z uwzględnieniem korelacji elektronowych (człon wymienny).

Na orbitale jednoelektronowe otrzymujemy równania:
\begin{equation}
\left[
-\frac{\hbar^2}{2m_e} \nabla + V_{ext}\left(\boldsymbol{r}\right) +
\underset{E_\text{Hartree}}{\underbrace{
\sum_{j=1}^{N_e} e^2 \int_{\mathbb{R}^3} \frac{\psi^*_j\left(\boldsymbol{r}'\right) \psi_j\left(\boldsymbol{r}'\right)}{\vert \boldsymbol{r} - \boldsymbol{r}' \vert} d^3r'}} -
\underset{E_\text{Exchange}}{\underbrace{
\sum_{j=1}^{N_e} e^2 \int_{\mathbb{R}^3} \frac{\psi_j\left(\boldsymbol{r}\right)}{\psi_i\left(\boldsymbol{r}\right)} \frac{\psi_j\left(\boldsymbol{r}'\right) \psi_i\left(\boldsymbol{r}'\right)}{\vert \boldsymbol{r} - \boldsymbol{r}' \vert} d^3r'		
	}}
\right]
\psi_i\left(\boldsymbol{r}\right) = \varepsilon_i \psi_i \left(\boldsymbol{r}\right)
\end{equation}
Obliczanie metodą self-consistent field SCF.\\
Potem można wstawić do w. Slatera i stąd policzyć gęstość elektronową.

Przybliżenie Hartree -> bez członu wymiennego.\\
Hartree-Focka-Bogoliubowa\\
\newpage


\section{Twierdzenia Hohenberga-Kohna}
\begin{thm}
\textbf{Hohenberga-Kohna}

Istnieje jednoznaczne odwzorowanie pomiędzy potencjał zewnętrznym
$\hat{V}_{ext}=\sum_{i=1}^{N}v_{ext}(\boldsymbol{r})$ a gęstością
cząstek $n_{0}\left(\boldsymbol{r}\right)$ w niezdegenerowanym stanie
podstawowym i tym samym całkowita energia układu jest wtedy jednoznacznym
funkcjonałem gęstości $n_{0}\left(\boldsymbol{r}\right)$ . \end{thm}
\begin{proof}
(\textbf{1. twierdzenia H-K})

Załóżmy, że istnieją dwa potencjały $\hat{V}_{ext}^{(1)}$ i $\hat{V}_{ext}^{(2)}$,
którym odpowiada pojedyncza gęstość cząstek w stanie podstawowym $n_{0}\left(\boldsymbol{r}\right)$
.

Stan $\ket{\Psi^{(1)}}$ odpowiadający $n_{0}\left(\boldsymbol{r}\right)$
jest stanem podstawowym z $\hat{V}_{ext}^{(1)}$, toteż z zasady wariacyjnej
wynika:

\[
\bra{\Psi^{(1)}}\hat{H}^{(1)}\ket{\Psi^{(1)}}<\bra{\Psi^{(1)}}\hat{H}^{(2)}\ket{\Psi^{(1)}}=\bra{\Psi^{(1)}}\hat{H}^{(1)}\ket{\Psi^{(1)}}+\bra{\Psi^{(1)}}\hat{H}^{(2)}-\hat{H}^{(1)}\ket{\Psi^{(1)}}
\]


I analogicznie stan $\ket{\Psi^{(2)}}$ odpowiadający $n_{0}\left(\boldsymbol{r}\right)$
jest stanem podstawowym z $\hat{V}_{ext}^{(2)}$, toteż z zasady wariacyjnej
wynika:

\[
\bra{\Psi^{(2)}}\hat{H}^{(2)}\ket{\Psi^{(2)}}<\bra{\Psi^{(2)}}\hat{H}^{(1)}\ket{\Psi^{(2)}}=\bra{\Psi^{(2)}}\hat{H}^{(2)}\ket{\Psi^{(2)}}+\bra{\Psi^{(2)}}\hat{H}^{(2)}-\hat{H}^{(1)}\ket{\Psi^{(2)}}
\]


Hamiltoniany $\hat{H}^{(1)}$ i $\hat{H}^{(2)}$ różnią się tylko
zewnętrznym potencjałem:

\[
\bra{\Psi^{(1)}}\hat{H}^{(1)}\ket{\Psi^{(1)}}<\bra{\Psi^{(1)}}\hat{H}^{(1)}\ket{\Psi^{(1)}}+\int_{\mathbb{R}^{3}}\left(V_{ext}^{(2)}\left(\boldsymbol{r}\right)-V_{ext}^{(1)}\left(\boldsymbol{r}\right)\right)n_{0}\left(\boldsymbol{r}\right)d^{3}r
\]
\[
\bra{\Psi^{(2)}}\hat{H}^{(2)}\ket{\Psi^{(2)}}<\bra{\Psi^{(2)}}\hat{H}^{(2)}\ket{\Psi^{(2)}}+\int_{\mathbb{R}^{3}}\left(V_{ext}^{(1)}\left(\boldsymbol{r}\right)-V_{ext}^{(2)}\left(\boldsymbol{r}\right)\right)n_{0}\left(\boldsymbol{r}\right)d^{3}r
\]


Dodając stronami:

\[
\bra{\Psi^{(1)}}\hat{H}^{(1)}\ket{\Psi^{(1)}}+\bra{\Psi^{(2)}}\hat{H}^{(2)}\ket{\Psi^{(2)}}<\bra{\Psi^{(1)}}\hat{H}^{(1)}\ket{\Psi^{(1)}}+\bra{\Psi^{(2)}}\hat{H}^{(2)}\ket{\Psi^{(2)}}
\]


lub 
\[
0<\int_{\mathbb{R}^{3}}\left(V_{ext}^{(2)}\left(\boldsymbol{r}\right)-V_{ext}^{(1)}\left(\boldsymbol{r}\right)+V_{ext}^{(1)}\left(\boldsymbol{r}\right)-V_{ext}^{(2)}\left(\boldsymbol{r}\right)\right)n_{0}\left(\boldsymbol{r}\right)d^{3}r=0
\]


Co jest wewnętrznie sprzeczne!

Stąd wniosek, że początkowe założenie jest nieprawdziwe, więc musi
istnieć tylko jeden $\hat{V}_{ext}$ odpowiadający dokładnie jednej
$n_{0}\left(\boldsymbol{r}\right)$.\end{proof}
\begin{thm}
\textbf{Hohenberga-Kohna}

The functional that delivers the ground state energy of the system,
gives the lowest energy if and only if the input density is the true
ground state density.

\textbf{Uwaga}

Gęstość \textgreek{r} 0 minimalizująca całkowitą energię jest dokładną
gęstością stanu podstawowego. A więc, dla próbnej gęstości (nieujemnej
i całkującej się do N ) zachodzi $E\left[ \tilde{\rho} \right] \leq ­ E\left[ \rho_0 \right] = E_0$ .\end{thm}
\begin{proof}
2. twd. Hohenberga-Kohna

W niezdegenerowanym stanie podstawowym wartość oczekiwana dowolnego
operatora jest funkcjonałem gęstości cząstek minimalizującej energię
stanu podstawowego. $n({\bf r})$ determines $v_{\hbox{ext}}({\bf r})$,
$N$ and $v_{\hbox{ext}}({\bf r})$ determine $\hat{H}$ and therefore
$\Psi$. This ultimately means $\Psi$ is a functional of $n({\bf r})$,
and so the expectation value of $\hat{F}$ is also a functional of
$n({\bf r})$, i.e. ${\displaystyle F[n({\bf r})]=\langle\psi\vert\hat{F}\vert\psi\rangle\,.}$

A density that is the ground-state of some external potential is known
as $v$-representable. Following from this, a $v$-representable energy
functional $E_{v}[n({\bf r})]$ can be defined in which the external
potential $v({\bf r})$ is unrelated to another density $n'({\bf r})$,
\[
E_{v}[n({\bf r})]=\int n'({\bf r})\,v_{\hbox{ext}}({\bf r})\,d{\bf r}+F[n'({\bf r})]\,
\]


and the variational principle asserts, 
\[
{\displaystyle \langle\psi'\vert\hat{F}\vert\psi'\rangle+\langle\psi'\vert\psi\rangle+\langle\psi\vert\hat{V}_{\hbox{ext}}\vert\psi\rangle\thinspace,}
\]
 

where $\psi$ is the wavefunction associated with the correct groundstate
$n({\bf r})$. This leads to, 
\[
{\displaystyle \int n'({\bf r})\,v_{\hbox{ext}}({\bf r})\,d{\bf r}+F[n'(......;\int n({\bf r})\,v_{\hbox{ext}}({\bf r})\,d{\bf r}+F[n({\bf r})]\,,}
\]


and so the variational principle of the second Hohenberg-Kohn theorem
is obtained, 
\[
\ensuremath{{\displaystyle E_{v}[n'({\bf r})]\,>\,E_{v}[n({\bf r})]\,.}}
\]


Although the Hohenberg-Kohn theorems are extremely powerful, they
do not offer a way of computing the ground-state density of a system
in practice. About one year after the seminal DFT paper by Hohenberg
and Kohn, Kohn and Sham {[}9{]} devised a simple method for carrying-out
DFT calculations, that retains the exact nature of DFT. This method
is described next. 
\end{proof}

\section{Metoda Kohna-Shama}


\subsection{Założenia}

Założenia w metodzie Kohna-Shama:
\begin{enumerate}
\item Zastępujemy układ oddziałujacych cząstek w zewnętrznym potencjale
$v_{ext}\left({\bf r}\right)$ układem pomocniczym składającym się
z quasicząstek nieoddziałujących w pewnym efektywnym potencjale $v_{eff}\left({\bf r}\right)$.
\item Zakładamy, że istnieje takie $v_{eff}\left({\bf r}\right)$, które
dokładnie odwzorowuje energię stanu podstawowego układu oddziałującego.
\item Układ nieoddziałujących cząstek jest opisany za pomocą orbitali Kohna-Shama
(pseudofunkcji falowych) $\phi_{i}\left({\bf r}\right)$ (są to rozw.
równania Kohna-Shama).
\end{enumerate}
Przy powyższych założeniach gęstość stanu podstawowego ukł. oddziałującego
jest zadana przez
\[
n\left({\bf r}\right)=\sum_{i=1}^{N}\phi_{i}\left({\bf r}\right)
\]



\subsection{Funkcjonał gęstości energii Kohna-Shama}

Dla hamiltonianu układu wielu ciał:

Na mocy twierdzeń Hohenberga-Kohna można utworzyć funkcjonał gęstości:
\[
E_{HK}\left[n\right]=T\left[n\right]+\int_{\mathbb{R}^{3}}v_{ext}\left({\bf r}\right)d^{3}r+\int_{\mathbb{R}^{3}}v_{int}\left({\bf r}\right)n\left({\bf r}\right)d^{3}r
\]


Przy założeniach metody Kohna-Shama można ten potencjał przedstawić
w postaci:

\[
E_{KS}\left[n\right]=T_{S}\left[n\right]+\int_{\mathbb{R}^{3}}v_{ext}\left({\bf r}\right)n\left({\bf r}\right)d^{3}r+E_{H}\left[n\right]+E_{Ion-Ion}\left[n\right]+E_{XC}\left[n\right]
\]


$T_{S}\left[n\right]=\int_{\mathbb{R}^{3}}\sum_{i=1}^{N}\frac{\hbar^{2}}{2m}\vert\nabla\phi_{i}\vert d^{3}r$
- funkcjonał energii kinetycznej

$E_{H}\left[n\right]=\int_{\mathbb{R}^{3}}\cdot d^{3}r$ - funkcjonał
energii Hartree {[}UZUPELNIĆ{]}

$E_{II}\left[n\right]=\text{const}$ - funkcjonał energii oddziaływania
jon-jon

$E_{XC}\left[n\right]=T\left[n\right]+\langle\hat{V}_{int}\rangle-T_{S}\left[n\right]-E_{H}\left[n\right]$
- funkcjonał energii korelacyjno-wymiennej, w ogolności analityczna
postać nie jest znana (gdyby była, to metoda KS byłaby metodą dokładną).


\subsection{Równanie Kohna-Shama}


\subsection{Hartree-Fock vs Kohn-Sham}



\newpage

\section{Funkcjonały gęstości elektronowej}


\subsection{Energia wymienna i koreelacyjna}


\subsection{LDA}


\subsection{GW}


\subsection{LDA+U/GW+U}



\newpage

\section{Metoda pseudopotencjałów}

\subsection{Postawienie problemu}
~\\
Rdzenie atomowe mają często dużo niższe energie poziomów energetycznych, więc bez sensowne jest oddzielanie ich od jądra atomowego. Dodatkowo w głębokiej studni potencjału (potencjał Coulombowski rdzenia jest osobliwy w środku jądra), funkcja falowo szybko oscyluje i potrzebna jest bardzo duża baza fal plaskich (inne też), by ją prawdiłowo opisać.\\
Czy można znaleźć taki potencjał $V^{\text{pseudo}}_{\mu\ell}(r)$ by opisać układ elektronów z wyższych orbitali (pasm) oddziałujących z rdzeniami atomowymi (jądro + głęboko położone elektrony), a by jakościowo oraz ilościowo odtwarzał problem dla wszystkich elektronów?\\
\begin{subequations}
\begin{equation} \label{eq:pseudo1}
\left[ -\frac{1}{2} \frac{1}{r} \frac{d^2}{d^2r} r + \frac{1}{2} \frac{\ell(\ell-1)}{r^2} + V^{\text{all e}}_{\mu\ell}(r)\right] R^{\text{all e}}_{\mu \ell}(r) = \left(\varepsilon_{\ell} + \delta \varepsilon_{\ell}\right) R^{\text{all e}}_{\mu \ell}(r)
\end{equation}
\begin{equation} \label{eq:pseudo2}
\left[ -\frac{1}{2} \frac{1}{r} \frac{d^2}{d^2r} r + \frac{1}{2} \frac{\ell(\ell-1)}{r^2} + V^{\text{pseudo}}_{\mu\ell}(r)\right] R^{\text{pseudo}}_{\mu \ell}(r) = \left(\varepsilon_{\ell} + \delta \varepsilon_{\ell}\right) R^{\text{pseudo}}_{\mu \ell}(r)
\end{equation}
\end{subequations}

\subsection{Warunki na pseudopotencjał}
~\\
\begin{enumerate}
	\item Dla pewnej („referencyjnej”) konfiguracji, wartości własne (energie) funkcji AE $R^{\text{all e}}_{\mu \ell}(r)$ i pseudo $ R^{\text{pseudo}}_{\mu \ell}(r)$ muszą się zgadzać.
	\item Funkcje falowe muszą być równe powyzej $r>r_{cut}$
	\item Pochodna logarytmiczna obu funkcji musi być równa dla $r_c$ 
	$$
	r_c\frac{{R^{\text{pseudo}}_{\mu \ell}}^{'}(r_c)}{R^{\text{pseudo}}_{\mu \ell}(r_c)} =
	\boxed{
	r_c \frac{d}{dr} \ln{R^{\text{pseudo}}_{\mu \ell}(r_c)} = 
	r_c \frac{d}{dr} \ln{R^{\text{all e}}_{\mu \ell}(r_c)} = 
	}
	r_c\frac{{R^{\text{all e}}_{\mu \ell}}^{'}(r_c)}{R^{\text{all e}}_{\mu \ell}(r_c)}
	$$
	\item Całkowity ładunek wewnątrz sfery $r_c$ musi być zachowany
	$$ \boxed{
	Q = e^2 \int_{0}^{r_c} r^2 R^{\text{all e}}_{\mu \ell}(r) =
	e^2 \int_{0}^{r_c} r^2 R^{\text{pseudo}}_{\mu \ell}(r) }
	$$
\end{enumerate}
\newpage

\section{Metody obliczania struktury pasmowej / Rozwiązywania równań Kohna-Shama}


\subsection{PW}
~\\
Rozwiązywanie równań w bazie fal płaskich:
\begin{equation}
\bra{\boldsymbol{r}}\ket{\boldsymbol{k}+\boldsymbol{G}} =
\frac{1}{\sqrt{V}} e^{i\left(\boldsymbol{k}+\boldsymbol{G}\right) \cdot \boldsymbol{r}}
\end{equation}
gdzie $\boldsymbol{G}$ wektor sieci odwrotnej spefłniający warunek $\frac{\hbar^2}{2m_e}\vert\boldsymbol{k}+\boldsymbol{G}\vert^2 \leq E_{\text{cut}}$
\subsection{OPW}
~\\


\subsection{APW}
~\\


\subsection{LCAO}
~\\
Szukamy rozwiązania w kostaci funkcji Blocha.
$$
\Psi_{n\boldsymbol{k}}\left(\boldsymbol{r}\right) = 
\frac{1}{\sqrt{N}} \sum_{\mu} c_{n\mu}\left(\boldsymbol{k}\right) \Phi_{\boldsymbol{k}\mu}\left(\boldsymbol{r}\right)
$$
Funkcja LCAO:
$$
\Phi_{\boldsymbol{k}\mu}\left(\boldsymbol{r}\right) = \frac{1}{\sqrt{N}} \sum_j e^{i\boldsymbol{k}\cdot\boldsymbol{R}_j} R_{\mu l}(\vert\boldsymbol{r}-\boldsymbol{R}_j\vert) Y_{lm}\left(\theta,\phi\right)
$$
Razem funkcję falową rozwijamy w bazie:
$$
\Psi_{n\boldsymbol{k}}\left(\boldsymbol{r}\right) = \frac{1}{N} \sum_{\mu} \sum_{j}
c_{n\mu}\left(\boldsymbol{k}\right) e^{i\boldsymbol{k}\cdot\boldsymbol{r}}
\underset{\text{orbitale atomowe}}{\underbrace{
		R_{\mu l}(\vert\boldsymbol{r}-\boldsymbol{R}_j\vert) Y_{lm}\left(\theta,\phi\right)
}}
$$
Funkcje radialne spełniają równanie:
$$
\left[ -\frac{1}{2} \frac{1}{r} \frac{d^2}{d^2r} r + \frac{1}{2} \frac{\ell(\ell-1)}{r^2} + V_{\mu\ell}(r)\right] R_{\mu \ell}(r) = \left(\varepsilon_{\ell} + \delta \varepsilon_{\ell}\right) R_{\mu \ell}(r)
$$
Dla każdego orbitalu musimy określić zasięg bazy, co można zrobić wprowadzając przesunięcie energii $\delta \varepsilon_{\ell}$.

Otrzymana baza nie jest ortonormalna, nie mniej jednak możemy rozwiązać metodą Ritza:
\begin{equation}
\sum_{\nu} H_{\mu\nu} c_{n\mu}\left(\boldsymbol{k}\right) = E_{n}\left(\boldsymbol{k}\right) \sum_{\mu} S_{\mu\nu} c_{n\mu}\left(\boldsymbol{k}\right)
\end{equation}
$\nu$ indeksuje składowe wektorów własnych.

Konstrukcja bazy LCAO:
\begin{itemize}
	\item Minimalny rozmiar bazy – \textbf{„single-$\zeta$” (SZ)} – jedna funkcja radialna na jedną zapełnioną powłokę ze względu na moment pędu
	\item Możliwości zwiększenia rozmiaru bazy – zwiększenie ilości funkcji radialnych \textbf{„multiple-$\zeta$” (DZ,DZP)} lub dodanie orbitali o różnym $\ell$ („polarization”)
\end{itemize}


\newpage
\subsection{Porównanie metod}
~\\
\begin{multicols}{2}
PW:

APW:
\begin{itemize}
	\item połączenie metod PW i LCAO
	\item 
\end{itemize}

OPW:

LCAO:
\begin{itemize}
	\item Używają pseudopotencjałów
	\item Bazą funkcji falowej jest zbiór orbitali atomowych
	\item Opis „zlokalizowany” – w duchu chemii kwantowej
	\item Wysoka wydajność
	\item Problemy ze zbieżnością
	\item Główne cechy bazy: wielkość, zakres orbitali, kształt
	\item Numeryczne orbitale atomowe (NAO) to rozwiązania numeryczne zagadnienia K-S dla izolowanego (pseudo-) atomu
\end{itemize}

\end{multicols}


\newpage



\section{Dynamika molekularna w ujęciu Ab-Initio}


\subsection{Twierdzenie Hellmana-Feynmanna}

\begin{thm} \label{theorem:HellmanFeynmann}
	Twierdzenie Feynmana-Hellmanna: pochodna energii
	całkowitej po pewnym parametrze jest równa wartości
	średniej pochodnej hamiltonianu po tym samym
	parametrze:
	$$ \frac{\partial E}{\partial \lambda} = \bra{\Psi} \frac{\partial}{\partial \lambda} \hat{H} \ket{\Psi} $$
\end{thm}
\begin{proof}
	???
\end{proof}

\subsection{Obliczanie sił działających na molekuły i relaksacja kryształu}
~\\
Na mocy twd. \ref{theorem:HellmanFeynmann} możemy napisać, że i-ta składowa siły jest równa:
\begin{subequations}
\begin{equation}
	F_i = -\frac{\partial E}{\partial X_i} = -\bra{\Psi} \frac{\partial}{\partial X_i} \hat{H} \ket{\Psi}
\end{equation}
\begin{equation}
	\boldsymbol{F} = -\nabla E = \int_{\mathbb{R}^3}\Psi^*\left(\boldsymbol{r}\right) \nabla H\left(\boldsymbol{r}\right) \Psi\left(\boldsymbol{r}\right) d^3r
\end{equation}
\end{subequations}

Relaksacja kryształu polega na znalezieniu minimum sił. Jest potrzebna do wyznaczenia struktury pasmowej.

\subsection{Dynamika ab-initio vs klasyczna}
~\\
\begin{multicols}{2}
\textbf{Klasyczna}
\begin{itemize}
\item Stały potencjał
\item Brak elektronowych stopni swobody
\item Brak opisu reakcji chemicznych
\item Zakres do ok. 100 $\mathring{A}$
\item Skala czasowa – do ok. 10 ns
\end{itemize}
\textbf{Ab-initio}
\begin{itemize}
\item Potencjał samozgodny
\item Elektronowe stopnie swobody
\item Opis powstawania i zrywania wiązań chemicznych (reakcji chemicznych)
\item Zakres do ok. 20 $\mathring{A}$
\item Skala czasowa – do ok. 100 ps
\end{itemize}
\end{multicols}

\subsection{Dynamika ab-initio Car-Parinello}
~\\

\subsection{Dynamika ab-initio BO}
~\\

\subsection{Obliczanie relacji dyspersyjnych dla fononów}
~\\
Teoria odpowiedzi liniowej – fonony

\newpage



\section{Informacje praktyczne}

Mamy przejebane...
\newpage

\end{document}
