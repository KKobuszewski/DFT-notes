\section{Funkcjonały gęstości elektronowej}


\subsection{Energia wymienna i koreelacyjna}
~\\
Najczęściej rozkłada się ten człon na sumę:
$$ E_{xc}\left[n\right] = E_{c}\left[n\right] + E_{c}\left[n\right] $$

In density functional theory, we define the exchange energy as:
$$ E_x \left[n\right] = \bra{\Phi[n]} \hat{V}_{e-e} \ket{\Phi[n]} - E_{\text{Hartree}} \left[n\right]$$ 
Czym jest stan $\ket{\Phi[n]}$???


\subsection{LDA}
~\\

\begin{equation}
E_x^{LDA} = -\frac{3}{4}\left(\frac{3}{\pi}\right)^{1/3}\int_{\mathbb{R}^3} n^{4/3}\left(\boldsymbol{r}\right) d^3r
\end{equation}


\subsection{GGA}
~\\


\subsection{GW}
~\\
Uwzględnienie ekranowania – metoda GW\\
• Ciało stałe – układ wieloelektronowy\\
Człon wymienno-korelacyjny\\
Oddziaływanie kulombowskie\\
Funkcja Greena/propagator:\\

$$ G() = $$

\subsection{LDA+U/GW+U}
~\\


