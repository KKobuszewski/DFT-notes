\section{Wielociałowa funkcja falowa}

Wielociałowa funkcja falowa

Katastrofa van Vlecka

Gęstość elektronowa

Hamiltonian układu w problemach ciała stałego:

\[
\hat{H}=\hat{T}_{e}\left(\nabla_{\boldsymbol{r}_{i}}\right)+\hat{T}_{jon}\left(\nabla_{\boldsymbol{R}_{i}}\right)+\hat{V}_{ext}\left(\boldsymbol{r}_{i},\boldsymbol{R}_{j}\right)+\hat{V}_{jon-jon}\left(\boldsymbol{R}_{i}-\boldsymbol{R}_{j}\right)+\hat{V}_{e-jon}\left(\boldsymbol{r}_{i}-\boldsymbol{R}_{j}\right)+\hat{V}_{e-jon}\left(\boldsymbol{r}_{i}-\boldsymbol{r}_{j}\right)
\]


Rozkłada się na energię kinetyczną elektronów, jonów (najczęściej
jąder/rdzeni atomowych), jednocząstkową energię potencjalną działającą
na cząstki (np. grawitacja) oraz operatory dwucząstkowe przedstawiające
wzajenmne oddziaływania cząstek na siebie.

